\documentclass[letterpaper, 10pt]{article}\usepackage[]{graphicx}\usepackage[]{color}
%% maxwidth is the original width if it is less than linewidth
%% otherwise use linewidth (to make sure the graphics do not exceed the margin)
\makeatletter
\def\maxwidth{ %
  \ifdim\Gin@nat@width>\linewidth
    \linewidth
  \else
    \Gin@nat@width
  \fi
}
\makeatother

\definecolor{fgcolor}{rgb}{0.345, 0.345, 0.345}
\newcommand{\hlnum}[1]{\textcolor[rgb]{0.686,0.059,0.569}{#1}}%
\newcommand{\hlstr}[1]{\textcolor[rgb]{0.192,0.494,0.8}{#1}}%
\newcommand{\hlcom}[1]{\textcolor[rgb]{0.678,0.584,0.686}{\textit{#1}}}%
\newcommand{\hlopt}[1]{\textcolor[rgb]{0,0,0}{#1}}%
\newcommand{\hlstd}[1]{\textcolor[rgb]{0.345,0.345,0.345}{#1}}%
\newcommand{\hlkwa}[1]{\textcolor[rgb]{0.161,0.373,0.58}{\textbf{#1}}}%
\newcommand{\hlkwb}[1]{\textcolor[rgb]{0.69,0.353,0.396}{#1}}%
\newcommand{\hlkwc}[1]{\textcolor[rgb]{0.333,0.667,0.333}{#1}}%
\newcommand{\hlkwd}[1]{\textcolor[rgb]{0.737,0.353,0.396}{\textbf{#1}}}%

\usepackage{framed}
\makeatletter
\newenvironment{kframe}{%
 \def\at@end@of@kframe{}%
 \ifinner\ifhmode%
  \def\at@end@of@kframe{\end{minipage}}%
  \begin{minipage}{\columnwidth}%
 \fi\fi%
 \def\FrameCommand##1{\hskip\@totalleftmargin \hskip-\fboxsep
 \colorbox{shadecolor}{##1}\hskip-\fboxsep
     % There is no \\@totalrightmargin, so:
     \hskip-\linewidth \hskip-\@totalleftmargin \hskip\columnwidth}%
 \MakeFramed {\advance\hsize-\width
   \@totalleftmargin\z@ \linewidth\hsize
   \@setminipage}}%
 {\par\unskip\endMakeFramed%
 \at@end@of@kframe}
\makeatother

\definecolor{shadecolor}{rgb}{.97, .97, .97}
\definecolor{messagecolor}{rgb}{0, 0, 0}
\definecolor{warningcolor}{rgb}{1, 0, 1}
\definecolor{errorcolor}{rgb}{1, 0, 0}
\newenvironment{knitrout}{}{} % an empty environment to be redefined in TeX

\usepackage{alltt}


\usepackage{parskip,xspace}
\usepackage{amsmath,amsthm,amsfonts,amssymb} 
\usepackage{caption}
\usepackage{xcolor} 
\usepackage{geometry}
\usepackage{fancyhdr}
\usepackage{multirow}
\usepackage{makecell}
\usepackage{ltxtable}
\usepackage{hyperref}
\usepackage{graphicx}
\usepackage{subfigure}
\usepackage{bm}

\geometry{left=2.5cm,right=2.5cm,top=2.5cm,bottom=2.5cm}


\newcommand{\ba}{$$\begin{aligned}}
\newcommand{\ea}{\end{aligned}$$}
\newcommand{\dd}{\mathrm{d}}
\allowdisplaybreaks





\pagestyle{fancy}
\lhead{Peng Shao 14221765}
\chead{}
\rhead{\bfseries STAT 8640 Fall 2015 Assignment 1}
\renewcommand{\headrulewidth}{0.4 pt}
\setlength{\parindent}{2em}
\IfFileExists{upquote.sty}{\usepackage{upquote}}{}
\begin{document}
\title{STAT 8640 Fall 2015 Assignment 1}
\author{Peng Shao 14221765}
\maketitle
\indent







$\blacktriangleright$ \textbf{Exercises 2.5.\quad Solution.} 

(1). \begin{itemize}
\item advantage: can fit many different functional forms; low bias; usually predict more accurately
\item disadvantage: overfitting problem; sually hard to interpret; high variance
\end{itemize}

(2). If our goal is to predict more accurately, it will usually be best to choose a more flexible approach.

(3). If our goal is to make some inferences, we prefer choosing a less flexible approach because the relation between response and predictor is more explicit.

$\blacktriangleright$ \textbf{Exercises 2.6.\quad Solution.} 

(1). The essential difference between parametric and non-parametric approach is that, the parametric make an assumption of the form of $f$, which can reduce problem of estimating $f$ down to one of estimating a set of parameter, but non-parametric do not make explicit assumptions about the functional form of $f$.


(2). \begin{itemize}
\item advantage: it is easier to estimate parameter; the relation between response and predictor is more explicit; 
\item disadvantage: the model we choose will usually not match the true unknown form of $f$; sometimes need more assumption.
\end{itemize}






$\blacktriangleright$ \textbf{Exercises 2.10.\quad Solution.} 



$\blacktriangleright$ \textbf{Exercises 3.5.\quad Solution.} 






$\blacktriangleright$ \textbf{Exercises 3.15.\quad Solution.}






$\blacktriangleright$ \textbf{Exercises 4.3.\quad Solution.}





$\blacktriangleright$ \textbf{Exercises 4.10.\quad Solution.} 

We know that we classify $X$ into $k$th class based on Bayes' classifier if
\ba
{p_k (x)}=\frac{f_k(x)\pi_k}{\sum_{l=1}^{K}f_l(x)\pi_l}
\ea
is largest among all $p_l(x),\, l=1,2,...,K$. For 1 dimension, the density of $x$ from $k$th class is 
$$
{f_k(x)}={\frac{1}{\sqrt{2\pi}\sigma_k}}e^{-\frac{(x-\mu_k)^2}{2\sigma_k^2}} 
$$
In comparing two classes $k$ and $l$, it is sufficient to look at the log-ratio, and we see that
\ba
\log\left(\frac{p_k(x)}{p_l(x)}\right)&=\log\left(\frac{\pi_k}{\pi_l}\right)
	\ea





	$\blacktriangleright$ \textbf{Exercises 4.13.\quad Solution.} 




	When you click the **Knit** button a document will be generated that includes both content as well as the output of any embedded R code chunks within the document. You can embed an R code chunk like this:

	```{r}
	summary(cars)
	```

	You can also embed plots, for example:

	```{r, echo=FALSE}
	plot(cars)
	```

	Note that the `echo = FALSE` parameter was added to the code chunk to prevent printing of the R code that generated the plot.
	s

	s

	s

	s

	s

	s

	s

	s

	s

	s

	s

	s

	s

	s

	s

	s

	s

	s

	s

	s
